\documentclass[11pt]{article}

\begin{document}

\begin{center}
Discrete Math 2311
\end{center}

\begin{center}
Proffesor Suat Sean Namli\\
\end{center}

\begin{center}
Mustafa Bedir Tapkan\\
Omer Faruk Yolal\\
Nadide Pasali\\
\end{center}

\newpage

\textbf{Starting Question: WRITE A COMPUTER PROGRAM} 
\textbf{BALL DICTIONARY}

\textbf{Problem:} There are 11 balls on a bench. All the balls are marked with a letter. 
Starting from the left side the letters are 'M', 'I', 'S', 'S', 'I', 'S', 'S', 'I', 'P', 'P', 'I'. We want to 
create a dictionary with all the possible words that we can create with these balls. %Create a dictionary using all the words that can be created from the letters of "Mississippi".
 \textbf{PS:} Dictionary is alphabetically ordered, and no words repeat the other. %Bedir- Done

\textbf{Extra:} Make it user interactive, create the dictionary based on the input.
\\

\textbf{Question 1:} There are seven \textbf{distinct} balls in box. Three of the balls are colored with red, the other four with blue. In how many ways can we put in order 

A) If there are no restrictions?

B) If the colors should alternate?

C) If all the red balls must be next to each other?

D) If all the red balls must be next to each other and all the blue balls must be next to each other?
\\
%A computer science professor has seven \textbf{different} programming 
%books on a bookshelf. Three of the books deal with C++, the other four with Java. In how 
%many ways can the professor arrange these books on the shelf\\  
%A) If there are no restrictions?
%B) If the languages should alternate?
%C) If all the C++ books must be next to each other?
%D) If all the C++ books must be next to each other and all the Java books must be next to each other?
%Nadide - done

\textbf{Question 2:}  A) How many different paths are there from $(-1,2,0)$ to $(1,3,7)$ in Euclidean three-space if each move is one of the following types? %
$$(H):(x,y,z)\rightarrow(x+1,y,z);$$
$$(V):(x,y,z)\rightarrow(x,y+1,z);$$
$$(A):(x,y,z)\rightarrow(x,y,z+1);$$

B) How many such paths are there from $(1,0,5)$ to $(8,1,7)$?

C) Generalize the results in parts $(A)$ and $(B)$.
\\

\textbf{Question 3:} A) There are two boxes, and 10 identical balls. Student will throw the 
ball into the boxes. In every throw he is able to put the ball inside one of the boxes. There 
is no rule for him to place the ball in which box. In how many ways can he throw the balls 
inside of the boxes %In how many possible ways could a student answer a 10-question true-false test? 

B) In how many ways can seven different colored ball be arranged about a circular table? 
\\ %Bedir

\textbf{Question 4:}  
\\
% Write a computer program (or develop an algorithm) to determine whether there is a three-digit integer $abc (= 100a + 10b + c)$ where $abc = a! + b! + c!$.
%Nadide - ?

\textbf{A word to the wise!} When dealing with any counting problem, we should ask 
ourselves about the importance of the order. If the order is relevance, we think in terms of 
permutations and arrangements and the rule of product. When order is not relevant, 
combination could play a key role in solving the problem.
\\

\begin{center}
\textbf{We need to be able to distinguish}

\textbf{Permutation from Combination}
\end{center}


\begin{center}
\textbf{Reminder}
\end{center}

$$\sum\limits_{i = m}^{m+n}{a_i = a_m+a_{m+1}+a_{m+2}+...+a_{m+n}}$$
$$\sum\limits_{i = 1}^5{a_i = a_1+a_2+a_3+a_4+a_5}$$
$$\sum\limits_{i = 1}^4{a_{i^2} = a_1+a_4+a_9+a_{16}}$$
$$\sum\limits_{i = 1}^4{\frac{1}{i} =\frac{1}{1}+\frac{1}{2}+\frac{1}{3}+\frac{1}{4}}$$
$$\sum\limits_{i = 1}^2{a_j = a_j+a_j}$$
\\

\textbf{Question:} In a highschool gym there are two groups of green balls and yellow balls, some of them has size 10, others are 11. There are 4 yellow and 6 green which has 10 size, 2 green and 3 yellow which has 11 size.\\
A. In how many ways we can choose a size 11 ball?\\
B. In how many ways we can choose a size 10 ball?\\
C. In how many ways we can choose a size 10 and 11ball?\\
\\


\textbf{Example} a) Suppose that Ellen draws five cards from a standard deck of 52 cards. 
In how many ways can her selection result in a hand with no clubs? Here we are 
interested in counting all five-card selections such as\\
i) ace of hearts, three of spades, four of spades, six of diamonds, and the jack of 
diamonds.\\
ii) five of spades, seven of spades, ten of spades, seven of diamonds, and the king of 
diamonds.\\
iii) two of diamonds, three of diamonds, six of diamonds, ten of diamonds, and the jack of 
diamonds.\\
If we examine this more closely we see that Ellen is restricted to selecting her five cards 
from the 39 cards in the deck that are not clubs. Consequently, she can make her 
selection in 
$\left( {\begin{array}{*{20}c}  39  \\  5  \\ \end{array}} \right)$ 
ways.\\
b) Now suppose we want to count the number of Ellen's five-card selections that contain 
at least one club. These are precisely the selections that were not counted in part (a). And 
since there are
$\left( {\begin{array}{*{20}c} 52  \\ 5  \\ \end{array}} \right)$ 
possible five-card hands in total, we find that 
$$ \left( {\begin{array}{*{20}c} 52  \\ 5  \\ \end{array}} \right) - \left( {\begin{array}{*{20}c} 39  \\ 5  \\ \end{array}} \right) = 2,598,960 - 575,757 = 2,023,203  $$
of all five-card hands contain at least one club.\\
c) Can we obtain the result in part (b) in another way? For example, since Ellen wants to 
have at least one club in the five-card hand, let her first select a club. This she can do in () 
ways. And now she doesn't care what comes up for the other four cards. So after she 
eliminates the one club chosen from her standard deck, she can then select the other four 
cards in GI.) ways. Therefore, by the rule of product, we count the number of selections 
here as 
$$ \left( {\begin{array}{*{20}c} 13  \\ 1  \\ \end{array}} \right) - \left( {\begin{array}{*{20}c} 51  \\ 4  \\ \end{array}} \right) = 13 \times 249,900 = 3,248,700  $$.
Something here is definitely wrong! This answer is larger than that in part (b) by more than one million hands. Did we make a mistake in part (b)? Or is something wrong with our present reasoning?

For example, suppose that Ellen first selects 
\begin{center}
the three of clubs
\end{center}
and then selects 
\begin{center}
the five of clubs,\\
king of clubs,\\
seven of hearts and,\\
 jack of spades. 
\end{center}
if, however, she first selects
\begin{center}
the five of clubs 
\end{center}
and then selects 
\begin{center}
the three of clubs, \\
king of clubs, \\
seven of hearts, \\
and jack of spades, \\
\end{center}
is her selection here really different from the prior selection we mentioned? Unfortu-nately, not And the case where she first selects 
\begin{center}
the king of clubs
\end{center}
and then follows this by selecting
\begin{center}
the three of clubs,\\
five of clubs,\\
seven of hearts, \\
and jack of spades \\
\end{center}
is not different from the other two selections mentioned earlier. Consequently, this approach is wrong because we are overcounting — by considering like selections as if they were distinct. 
\\

EDIT
\textbf{There is one more statement here if you want we can add}
\\

\begin{center}
1\\1 1\\1 2 1\\1 3 3 1\\1 4 6 4 1\\1 5 10 10 5 1\\1 6 15 20 15 6 1\\
\end{center}

\textbf{Theorem} - \textbf{The Binomial Theorem}
$$\left({x+y}\right)^n = \left( {\begin{array}{*{20}c} n  \\ 0  \\ \end{array}} \right) x^0 y^n +  \left( {\begin{array}{*{20}c} n  \\ 1  \\ \end{array}} \right) x^1 y^{n-1} + \left( {\begin{array}{*{20}c} n  \\ 2 \\ \end{array}} \right) x^2 y^{n-2} + ... + \left( {\begin{array}{*{20}c} n  \\ n \\ \end{array}} \right) x^n y^0$$

$$= \sum\limits_{k = 0}^n {\left( {\begin{array}{*{20}c} n  \\ k  \\ \end{array}} \right) x^k y^{n-k}}$$
\\

\textbf{Question 6: } Find the coefficient of $a^5b^2$ in the expansion of  $\left(2a-3b\right)^7$
\\ %Yolal - Done

\textbf{Question 7: }  $x+y = n$ , $x,y \geq 0 \in Z$. How many solutions do we have? 
\\ %Bedir

\textbf{Question 8: }  $x+y+z = n$ , $x,y,z \geq 0 \in Z$. How many solutions do we have?
\\ %Nadide - done

\textbf{Question 9: }  $x+y+z+t = n$ , $x,y,z,t \geq 0 \in Z$. How many solutions do we have?
\\ %Yolal - Done

\textbf{Formula: }  $$\left( {\begin{array}{*{20}c} n  \\ k  \\ \end{array}} \right) = \left( {\begin{array}{*{20}c} n+1  \\ k+1  \\ \end{array}} \right) - \left( {\begin{array}{*{20}c} n  \\ k+1  \\ \end{array}} \right)$$

\begin{center}
1\\1 1\\1 2 1\\1 3 3 1\\1 4 6 4 1\\1 5 10 10 5 1\\1 6 15 20 15 6 1\\
$\downarrow$\\
From Pascal Triangle\\
$\downarrow$
\end{center}

$$\left( {\begin{array}{*{20}c} n+2  \\ 2  \\ \end{array}} \right) + \left( {\begin{array}{*{20}c} n+1  \\ 2  \\ \end{array}} \right) + ... + \left( {\begin{array}{*{20}c} 2  \\ 2  \\ \end{array}} \right) = \left( {\begin{array}{*{20}c} n+3 \\ 3  \\ \end{array}} \right)$$
\begin{center}
Telescoping
\end{center}

\textbf{Formula: } $$x_1 + x_2 + x_3 + ... + x_r = n , x_1 \geq 0$$ $$= \left( {\begin{array}{*{20}c} n+r-1 \\ r-1 \\ \end{array}} \right)$$
\\

\textbf{Question 10: }  $a+b+c+d = 100$ , $a,b,c,d \geq 1$ How many different ways you can choose a,b,c,d?
\\ %Bedir

\textbf{Question 11: }  $a+b+c = 100$ , c is even , $a,b,c \geq 0$. How many different ways you can choose a,b,c?
\\ %Nadide - done

\textbf{Question 12: }  We have one dollar. We want to broke it to pennies and nickels. How many different ways can we do this operation?
\\ %Yolal Done

\textbf{Question 13: }  We have one dollar. We want to broke it to pennies, nickels, and dimes. How many different ways can we do this operation?
\\ %Bedir

IMPROVE VERBAL
\\

\textbf{Homework: } We have one dollar. We want to broke it to pennies, nickels, dimes, and quarters. How many different ways can we do this operation? Come up with a formula for it. $p+n+d+q$.
\\ %Nadide

\textbf{Question 14: }  There are 7 identical balls. How can you distribute these to 3 same ladies ?
\\ %Yolal Done

\textbf{Question 15: }  There are 7 orange balls and 6 yellow balls. How can you distribute these balls to 3 ladies ?
\\ %Bedir

\textbf{Question 16: }  There are 10 identical balls. How can you distribute these to 3 ladies and 4 gentleman ?
\\ %Nadide - done

\textbf{Question 17: }  There are 10 orange and 7 green balls. How can you distrubute these to 3 yellow and 4 blue bags ?
\\%Yolal Done

\textbf{Question 18: }  There are 10 yellow balls. How can you distribute these to 3 ladies and 4 gentleman with following conditions?
\par %Bedir
\textbf{A.} All ladies will get at least 1 ball.

\textbf{B.} Sum of the balls that all ladies get will be greater than sum of the balls that all gentlemen get?
\\

\textbf{Question 19: } $y_1 + ... + y_6 = 10$ , how many possibilities for $y$ values?
\\ %Nadide - done
 
\textbf{Question 20: } $0 \leq y_1 + ... + y_5 \leq 10$ , how many possibilities for $y$ values?
\\ %yolal - Done

\textbf{Formula: } $$\left( {\begin{array}{*{20}c} n \\ n \\ \end{array}} \right) +  \left( 
{\begin{array}{*{20}c} n+1 \\ n \\ \end{array}} \right) + ... +  \left( {\begin{array}{*{20}c} n+m 
\\ n \\ \end{array}} \right) =  \left( {\begin{array}{*{20}c} n+m+1 \\ n+1 \\ \end{array}} 
\right)$$

\newpage

\begin{center}
\textbf{ANSWERS}
\end{center}

\textbf{Question 1} \\

A) $7!$ \\

B) $1$ \\

C) $3! 5!$ \\

D) $3! 4! 2!$ \\

\textbf{Question 3} \\

A) $2*2*2*2*2*2*2*2*2*2 = 2^{10}$ \\

B) $(7-1)! = 6! = 720$ \\

\textbf{Question 6}
$$= \sum\limits_{k = 0}^n {\left( {\begin{array}{*{20}c} n  \\ k  \\ \end{array}} \right) x^k y^{n-k}} \Rightarrow \left(2a-3b\right)^7 \Rightarrow \left( {\begin{array}{*{20}c} 7 \\ 5 \\ \end{array}} \right)\left(2a\right)^5\left(-3b\right)^2 $$

$$ \left( {\begin{array}{*{20}c} 7 \\ 5 \\ \end{array}} \right)2^5\left(-3\right)^2 = \frac{7*6}{2} 2^5  9 = 42 * 16 * 9 = {6048} $$

\textbf{Question 7}

$$total = n-1$$

\textbf{Question 8 }

$$ {n+r-1 \choose r-1} \Rightarrow {n+3-1 \choose 3-1} = {n+2 \choose 2} $$

\textbf{Question 9 }

$$ {n+r-1 \choose r-1} \Rightarrow {n+4-1 \choose 4-1} = {n+3 \choose 3} $$ 

\textbf{Question 10}

$$\left(a-1\right)+\left(b-1\right)+\left(c-1\right)+\left(d-1\right) = 96 $$
$$\left(a-1\right),\left(b-1\right),\left(c-1\right),\left(d-1\right) \geq 0$$
$$x_1 + x_2 + x_3 + ... + x_r = n , x_1 \geq 0$$ 
$$\left( {\begin{array}{*{20}c} n+r-1 \\ r-1 \\ \end{array}} \right) \Rightarrow \left( {\begin{array}{*{20}c} 96+4-1 \\ 4-1 \\ \end{array}} \right) = \left( {\begin{array}{*{20}c} 99 \\ 3 \\ \end{array}} \right)$$

\textbf{Question 11}
 
 $$a+b+2c = 100$$

 $$c=0  \quad  a+b=100 \rightarrow 101 $$
 $$c=1  \quad  a+b=98  \rightarrow 99 $$
 $$c=2  \quad  a+b=96  \rightarrow 97 $$
 $$...$$
 $$c=50 \quad  a+b=0   \rightarrow 1  $$

 $$101+99+97+..+1 = 2601$$
 
 \textbf{Question 12}
 
 $$p+5n = 100$$
 
$$n=0  \quad p=100 \rightarrow 1$$
$$n=1  \quad p=95 \rightarrow 2$$
$$n=2  \quad p=90 \rightarrow 3$$
$$...$$
$$n=20 \quad p=0 \rightarrow 21$$

$21$ Different ways to broke 1 dollar into pennies and nickles

\textbf{Question 14}

$$l_1 + l_2 + l_3 = 7$$

$${n+r-1 \choose r-1} \Rightarrow {7+3-1 \choose 3-1} = {9 \choose 2}$$

\textbf{Question 16}

$$l_1 + l_2 + l_3 + g_1 + g_2 + g_3 + g_4 = 10$$

$${n+r-1 \choose r-1} \Rightarrow {10+7-1 \choose 7-1} = {16 \choose 6}$$ 

\textbf{Question 17}

$$y_1 + y_2 + y_3 + b_1 + b_2 + b_3 + b_4 = 10 + 7 $$(Since there is no condition for color)
 
$${n+r-1 \choose r-1} \Rightarrow {17+7-1 \choose 7-1} = {23 \choose 6}$$ 
 
\textbf{Question 19}

$${n+r-1 \choose r-1} \Rightarrow {10+6-1 \choose 6-1} = {15 \choose 5}$$ 
 
 \textbf{Question 20}
 
 $6^{th} \rightarrow y_1 + y_2 + y_3 + y_4 + y_5 + y_6 = 10$
 
 $y_6$ can be max $10$ and min $0$ which means $y_6 = y_1 + y_2 + y_3 + y_4 + y_5$ so
 
 $${n+r-1 \choose r-1} \Rightarrow {10+6-1 \choose 6-1} = {15 \choose 5}$$ 
 
\end{document}

